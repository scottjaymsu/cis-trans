\section{Methods}
\subsection{Modeling \emph{Pseudomonas aeruginosa} membrane}
The membrane lipidomic data was extracted from the work done by R{\"u}hl et al.\cite{ruhl2012glycerophospholipid}.
The headgroup composition of membrane was 12:7:1 for PE:PG:CL.
The detailed composition of the lipids along with tails is given in the table \ref{}  

\begin{table}
	\caption{\label{tab:bilayers}Head group and tail composition of one leaflet of \emph{Pseudomonas aeruginosa} bilayer used for molecular dynamics simulation. }
	
	\begin{tabular}{lrlr}
		\toprule
		Lipid Type &  Total Number & Name & Tails \\
		\midrule
		PE(34:2)   &   28		   & YOPE &  16:1/18:1 \\
		PE(32:1)   &   20		   & PYPE &  16:0/16:1 \\
		PE(34:1)   &   14		   & POPE &  16:0/18:1 \\
		PE(34:2)   &   28		   & YOPE &  16:1/18:1
		
	\end{tabular}
\end{table}

\subsection{Simulation Protocol}
The energy levels were minimized in 1000 steps using NAMD version 2.14\cite{phillips2005scalable}. Then, the simulations were carried out on NAMD version 3.0alpha9. Ten molecules were placed onto both sides of the membrane for each simulation. The simulation ran for 2000 nanoseconds.
The simulations were done using the CHARMM36 force field in NAMD 2.11\cite{phillips2005scalable}. Both the models were TIP3P water models with a 12 Angstrom nonbonded cutoff. Anisotropic pressure coupling was held consistent using the Langevin piston method at a pressure of 1 atm. Using the SETTLE algorithm to fix hydrogen bonds at the same length, the time step was at 2 fs.