\section{Introduction}

Living cells have evolved various mechanisms to acclimate to environmental changes. 
The ability of an organism to become accustomed to new ambient conditions plays a vital role in keeping the cells alive. 
Bacterial and eukaryotic cells apply different adaptive mechanisms under various environmental stresses. 
For example, heat shock and osmotic pressure influence eukaryotic cell membrane fluidity and gene expression\cite{los2004membrane}.
Bacteria respond to a new environmental condition using either a short-term or a long-term adaptation mechanism. 
Short-term responses occur immediately in the company of heat shocks or toxic substances\cite{eberlein2018immediate}.
Gram-negative bacteria are considered to show a short-term response after the incidence of environmental stress, and they isomerize unsaturated cis lipid tails into the trans ones. As a result, Gram-negative bacteria change their membrane composition by replacing lipid molecules containing unsaturated tails with saturated phospholipid tails\cite{eberlein2018immediate,heipieper1992conversion,okuyama1991cis}.
This alteration increases the mobility of phospholipids and leads to an increase in membrane fluidity. The above-mentioned process helps bacteria to survive rapid or severe environmental changes. 

\emph{Pseudomonas putida}, a Gram-negative bacterium, is resistant to harsh conditions like high pH or low nutrition\cite{poblete2012industrial,timmis2002pseudomonas}.
Due to its fast growth in simple bacterium culture, \emph{P.putida} has drawn researchers attention to its use in the industry, from synthesizing biopolymers to degrading xenobiotic substances. 
For instance, polyhydroxyalkanoates (PHA), a biocompatible polymer produced by \emph{P.putida}, has various applications in biodegradable packaging and tissue engineering\cite{poblete2012industrial}.
Besides polymer production, this bacterial strain is an excellent production host for other bacterial genes. As an example, \emph{P.putida} can produce some complex natural products of myxobacteria which are applied as high-values pharmaceuticals \cite{wenzel2005heterologous,gross2005posttranslational}. 
\emph{P. putida} is a potential exchange platform for antimicrobial resistance genes (ARGs) and can contribute to the spread of ARGs by some virulent bacteria like \emph{Pseudomonas aeruginosa}.\cite{peter2017genomic}
Because of its vast importance, this bacterial strain has been widely studied experimentally.\cite{tohya2022whole,martinez2022evaluation,wang2009production,kulkarni2006biodegradation,martinez2014pseudomonas} 

Different components, like organic solvents, can affect the membrane properties of bacteria. 
Organic solvents increase the fluidity of lipid bilayers and influence the activity and viability of bacteria. According to Carla et al., \emph{P. putida} cannot survive in the presence of organic solvents because the bacterial membrane is disturbed, and \emph{P. putida} cannot divide and develop new colonies.\cite{de2004mycobacterium}
Jana R{\"u}hl et al. identified up to 127 phospholipid species in the \emph{P. putida} membrane using the liquid chromatography/mass spectrometry (LC/MS) method.\cite{ruhl2012glycerophospholipid} 
They also investigated the bacterial membrane composition under different environmental conditions. The results demonstrated that \emph{P. Putida} could cope with the environmental stresses by changing the lipid head group composition and modifying the lipid acyl chain degrees of saturation. 
Therefore, the membrane composition and its properties play a critical role in bacterial cell viability and survival.

Bacteria have adapted many tools enabling their survival in the presence of environmental stresses. 
For environmental changes that occur suddenly, a quick response is necessary such as after heat shock or the presence of toxic organic solvents. 
One short-term response is an isomeration of cis into trans unsaturated fatty acids to increase the rigidity of a cell membrane\cite{heipieper1992conversion,okuyama1991cis}. 
Another response is the release of outer membrane vesicles which increases the hydrophobicity of cells\cite{schwechheimer2015outer,beveridge1999structures,kuehn2005bacterial,mashburn2008gram}.
These responses enable cells to stay physiologically active. In Gram-negative bacteria, the outer layer of the cell is the envelope, and it has an outer membrane with lipopolysaccharides facing the extracellular space and an inner membrane by the cytoplasm. The space between the inner membrane and the envelope is called periplasm, and it contains peptidoglycan. The inner membrane is the main barrier, and it is greatly affected by outside stressors like hydrocarbons and heat, which increase membrane fluidity\cite{beney2001influence,hazel1990role,heipieper1994adaptation}.
An excessively fluid membrane can break electrochemical gradients\cite{isken1998bacteria}.

Understanding the mechanism of action behind the cis-trans isomerization of the bacterial membrane is essential. 
Thus, in this work, we explored the impact of fatty acids cis-trans isomerization on the properties and dynamics of the \emph{P. putida} membrane model using molecular dynamics (MD) simulation. 
We used a membrane model derived from experimental studies \cite{isken1998bacteria} in two forms: membrane models containing all-cis and all-trans fatty acid conformations.


\newpage